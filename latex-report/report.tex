\documentclass[10pt,conference,compsocconf]{IEEEtran}

\usepackage{hyperref}
\usepackage{graphicx}	% For figure environment


\begin{document}
\title{Machine Learning - Higgs Boson Project}

\author{
  Lucien Michaël Iseli, Florian Maxime Charles Ravasi and Jules Eliot Gottraux\\
  \textit{Master of Data Science, EPFL, Switzerland}
}

\maketitle

\section{Introduction}

\section{Pipeline}

\begin{enumerate}
	\item{Copied 6 function we have to return from the labs}
	\item{Changed the functions we copied from the labs such that they always assume that vectors are represented in (N,1)}
	\item{Plotted distributions of features to gain insight}
	\item{Gradient descent on data to see what's going on}
	\item{Try to remove features based on the plots, the ones that look like they don't add anything}
	\item{Logistic regression, see if results make sense}
	\item{Try polynomial expansion to have better accuracy}
	\item{Try polynomial expansion with cross products to have better accuracy}
	\item{Realize that the weird distributions of many features with extreme variance are most likely due to the fact that -999 values are unknown values. Seems weird that many of the values are around zero, and many of them are exactly -999}
	\item{Normalize data without taking into account the -999 values, and set -999 values to 0. Such that they shouldn't affect the result as when 0 will be multiplied with the weight it'll be 0, i.e. not contribute.}
	\item{Replotted the distributions of the features after normalizing and removing the -999 values}
	\item{Notice that some plots look like uniform distribution or clearly don't give us insight on the result. => can remove them to have faster algorithm (NOT DONE YET)}
	\item{When plotting the distributions of features without unknown values we noticed someting very strange: all of the distributions are continuous except one! It only has 4 values}
	\item{We thought that maybe this value is some sort of category, thus maybe we should treat them differently => we trained them separatly, we separate them in 4 categories based on that feature and train them separatly}
	\item{Trained the model that way with linear regression (no expansion), obtained much better results}
	\item{Tried logistic regression but results weren't as good}
	\item{Tried to add feature expansion, square and sqrt (without cross products)}
\end{enumerate}

\end{document}
